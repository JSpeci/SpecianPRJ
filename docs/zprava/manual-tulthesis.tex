\documentclass[FM,RP]{tulthesis}
% tento dokument používá balíky specifické pro XeLaTeX a lze jej přeložit
% jen XeLaTeXem, nemáte-li instalována použitá (komerční) písma, změňte
% nebo vymažte příkazy \set...font na následujících řádcích

\newcommand{\verze}{1.7}

\usepackage{polyglossia}
\setdefaultlanguage{czech}
\usepackage{xevlna}

\usepackage{makeidx}
\makeindex

% fonty
\usepackage{fontspec}
\usepackage{xunicode}
\usepackage{xltxtra}
\usepackage{graphicx}
\usepackage{pdfpages} 


% příkazy specifické pro tento dokument
\newcommand{\argument}[1]{{\ttfamily\color{\tulcolor}#1}}
\newcommand{\argumentindex}[1]{\argument{#1}\index{#1}}
\newcommand{\prostredi}[1]{\argumentindex{#1}}
\newcommand{\prikazneindex}[1]{\argument{\textbackslash #1}}
\newcommand{\prikaz}[1]{\prikazneindex{#1}\index{#1@\textbackslash #1}}
\newenvironment{myquote}{\begin{list}{}{\setlength\leftmargin\parindent}\item[]}{\end{list}}
\newenvironment{listing}{\begin{myquote}\color{\tulcolor}}{\end{myquote}}
\sloppy

% deklarace pro titulní stránku

\TULtitle{Vytvoření výukové aplikace řešící blokové diagramy bezporuchovosti (RBD)}{}
\TULprogramme{B2646}{Informační technologie}{}
\TULbranch{1802R007}{Informační technologie}{}
\TULauthor{Jan Špecián}
\TULsupervisor{Ing. Josef Chudoba, Ph.D.}
\TULyear{2019}

\begin{document}
%\ThesisStart{pic/zadaniBPScan.pdf}
\ThesisStart{male}
%\includepdf[scale=1,angle=0,pages=-]{pic/prohlaseni.pdf} 

\begin{abstractCZ}
    Práce je zaměřena na tvorbu desktopové aplikace pro tvorbu RBD diagramů a spojených výpočtů a vyzualizací.
\end{abstractCZ}

\begin{keywordsCZ}
    RBD
\end{keywordsCZ}

\vspace{2cm}



\clearpage

\begin{acknowledgement}
    Tímto bych rád poděkoval Ing. Josef Chudobovi, Ph.D. za věnovaný čas v konzultacích a odborné vedení plné trpělivosti a s tím spojené nabyté zkušenosti.
\end{acknowledgement}

\tableofcontents
\listoffigures

\clearpage

\begin{abbrList}
    \textbf{JSON} & JavaScript Object Notation \\
    \textbf{LINQ} & Language Integrated Query \\
    %\textbf{PDF} & Portable Document Format \\
   
\end{abbrList}

\chapter*{Úvod}

\chapter*{Přehled existujících softwarových nástrojů}

\chapter*{Teoretický úvod}

\chapter*{Návrh desktopové aplikce .NET}

\chapter*{Průběh vývoje}

\chapter*{Testování}

\chapter*{Návod k použití}

\chapter*{Závěr}



\begin{thebibliography}{Mm99}
    \bibitem{1}
        ROTH, Daniel, Rick ANDERSON a Shaun LUTTIN, Introduction to ASP.NET Core [online]. Microsoft [cit. 2019-4-10]. Dostupné z: https://docs.microsoft.com/en-us/aspnet/core/?view=aspnetcore-2.2
    \bibitem{2}
        KURTZ, Jamie, 2013. ASP.NET MVC 4 and the Web API: building a REST service from start to finish. Berkeley, CA: Apress. Expert's voice in ASP.NET.
    
\end{thebibliography}

\end{document}
