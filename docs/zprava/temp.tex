\documentclass[FM,RP]{tulthesis}
% tento dokument používá balíky specifické pro XeLaTeX a lze jej přeložit
% jen XeLaTeXem, nemáte-li instalována použitá (komerční) písma, změňte
% nebo vymažte příkazy \set...font na následujících řádcích

\newcommand{\verze}{1.7}

\usepackage{polyglossia}
\setdefaultlanguage{czech}
\usepackage{xevlna}

\usepackage{makeidx}
\makeindex

% fonty
\usepackage{fontspec}
\usepackage{xunicode}
\usepackage{xltxtra}
\usepackage{graphicx}
\usepackage{pdfpages} 


% příkazy specifické pro tento dokument
\newcommand{\argument}[1]{{\ttfamily\color{\tulcolor}#1}}
\newcommand{\argumentindex}[1]{\argument{#1}\index{#1}}
\newcommand{\prostredi}[1]{\argumentindex{#1}}
\newcommand{\prikazneindex}[1]{\argument{\textbackslash #1}}
\newcommand{\prikaz}[1]{\prikazneindex{#1}\index{#1@\textbackslash #1}}
\newenvironment{myquote}{\begin{list}{}{\setlength\leftmargin\parindent}\item[]}{\end{list}}
\newenvironment{listing}{\begin{myquote}\color{\tulcolor}}{\end{myquote}}
\sloppy

% deklarace pro titulní stránku

\TULtitle{Vytvoření výukové aplikace řešící blokové diagramy bezporuchovosti (RBD)}{}
\TULprogramme{B2646}{Informační technologie}{}
\TULbranch{1802R007}{Informační technologie}{}
\TULauthor{Jan Špecián}
\TULsupervisor{Ing. Josef Chudoba, Ph.D.}
\TULyear{2019}

\begin{document}
%\ThesisStart{pic/zadaniBPScan.pdf}
\ThesisStart{male}
%\includepdf[scale=1,angle=0,pages=-]{pic/prohlaseni.pdf} 

\begin{abstractCZ}
    Práce je zaměřena na tvorbu desktopové aplikace pro tvorbu RBD diagramů a spojených výpočtů a vyzualizací.
\end{abstractCZ}

\begin{keywordsCZ}
    RBD
\end{keywordsCZ}

\vspace{2cm}



\clearpage

\begin{acknowledgement}
    Tímto bych rád poděkoval Ing. Josef Chudobovi, Ph.D. za věnovaný čas v konzultacích a odborné vedení plné trpělivosti a s tím spojené nabyté zkušenosti.
\end{acknowledgement}

\tableofcontents
\listoffigures

\clearpage

\begin{abbrList}
    \textbf{RBD} & Reliability Block Diagram \\
    %\textbf{JSON} & JavaScript Object Notation \\
    %\textbf{LINQ} & Language Integrated Query \\
    %\textbf{PDF} & Portable Document Format \\
   
\end{abbrList}

\chapter*{Úvod}
    U každého systému je velmi důležitá jeho funkční spolehlivost během doby jeho životnosti. Každý systém, pokud má existovat a fungovat co nejdéle a přitom bez závad,
    nebo alespoň s jejich co nejmenším počtem, musí splňovat jednu zásadní vlastnost, a tou je spolehlivost. 
    Požadavek na dostatečně velkou a často až maximální spolehlivost námi užívaných systémů má tudíž zcela zásadní význam z hlediska bezpečnostního, ekonomického i
    ekologického. 

    Cílem ročníkového projektu je navrhnout a implementovat desktopovou aplikaci pro tvorbu a jednoduchou vizualizaci RBD diagramů a výpočet parametrů spolehlivosti.
    Zobrazit střední dobu do poruchy pro každý blok a poskytnout možnost vizualizace distribuční funkce pro každý blok v kombinace sérriového a paralelního zapojení bloků.

\chapter{Přehled existujících softwarových nástrojů}

\chapter{Teoretický úvod}
        
    \section{Distribuční funkce spojité náhodné veličiny}
        Jedním z prostředků pro popis náhodné veličiny je distribuční funkce, která každému
        reálnému číslu přiřazuje pravděpodobnost, že náhodná veličina nabude hodnoty menší nebo
        rovné než toto číslo.\cite{6}
        U spojité náhodné veličiny se užívá k jejímu popisu distribuční funkce F(x) definované vztahem: 
        $$ F(x_{i}) = P(X<x_{i}) $$
            \begin{figure}[h]
                \centering
                \includegraphics[scale=0.75]{pic/distrib.png}
                \caption{Příklad průběhu distribuční funkce exponenciálního rozdělení} \label{Obrázek č. 2.1}
            \end{figure}
        \subsubsection{Vlastnosti distribuční funkce}
            \begin{itemize} 
                \item
                Hodnoty distribuční funkce leží v intervalu od nuly do jedné.
                $$ 0 \leq F(x) \leq 1 $$
                \item
                Distribuční funkce je neklesající.
                $$  P(x_{1} \leq X < x_{2}) = F(x_{2}) - F(x_{1})   pro x_{1} < x_{2} $$
                \item
                V záporném nekonečnu se blíží k nule, v kladném nekonečnu se blíží k jedné.
                $$ F(- ∞) = 0, F(∞) = 1 $$ 
            \end{itemize}

    \section{Exponenciální rozdělění}
        Toto rozdělení má spojitá náhodná veličina X, která představuje dobu čekání do nastoupení (poissonovského) náhodného jevu, 
        nebo délku intervalu (časového nebo délkového) mezi takovými dvěma jevy (např. doba čekání na obsluhu, vzdálenost mezi dvěma poškozenými místy na silnici, doba do poruchy).
        Závisí na parametru $ \lambda $, což je převrácená hodnota střední hodnoty doby čekání do nastoupení sledovaného jevu. \cite{7}

        Náhodná veličina X má exponenciální rozdělení Exp($ \lambda $) právě tehdy, když je hustota pravděpodobnosti dána vztahem:
        $$  f(x) = \left\{ \begin{array}{ll}
            0 & \mbox{pro }x<0 \\
            \lambda.e^{\lambda.x} & \mbox{pro }x\geq 1
            \end{array} \right. $$
            \begin{figure}[h]
                \centering
                \includegraphics[scale=0.75]{pic/hustota.png}
                \caption{Příklad grafu hustoty pravděpodobnosti exponenciálního rozdělení} \label{Obrázek č. 2.1}
            \end{figure}

        \subsubsection*{Další vlastnosti}
        \begin{itemize} 
        \item
            $  F(x) = \left\{ \begin{array}{ll}
                0 & \mbox{pro }x<0 \\
                1-e^{-\lambda.x} & \mbox{pro }x\geq 1
                \end{array} \right. $
        \item
            $ E(x) = \frac{1}{\lambda}  $
        \item
            $ D(x) = \frac{1}{\lambda^{2}}  $

        \end{itemize}

    \section{Spolehlivost a střední doba mezi poruchami}
        \subsubsection{Střední doba mezi poruchami}
            Základní veličinou pro měření spolehlivosti systému je střední doba mezi poruchami (MTBF, Mean Time
            Between Failure). Obvykle je udávána v hodinách. Čím vyšší je hodnota MTBF, tím vyšší je spolehlivost
            produktu.\cite{3}
            Je statistická veličina používaná ke kvantifikaci spolehlivosti součásti, či celého výrobku.
            Určuje se pro výrobek nebo zařízení, které se opravuje. \cite{3}
        \subsubsection{Spolehlivost}
            Spolehlivost je schopnost systému nebo součásti vykonávat požadované funkce za daných
            podmínek po určené časové období \cite{4} 

            $$ Spolehlivost = e^{-(\frac{Doba}{MTBF})}$$
    
    \section{Analýza blokového diagramu bezporuchovosti (RBD)}

        Analýza blokového diagramu bezporuchovosti (RBD - Reliability Block Diagram) je metoda analýzy systému. 
        Diagram RBD je grafická reprezentace logické struktury systému v podobě podsystémů a/nebo součástí. 
        To umožňuje, aby byly cesty úspěchu (funkceschopného stavu) reprezentovány tak, jak jsou bloky (podsystémy/součásti) logicky propojeny.\cite{1}

        Blokové diagramy jsou mezi prvními úkoly dokončenými během etapy vymezení produktu. 
        Mají být vypracovány jako součást vývoje počáteční koncepce. 
        Práce na nich mají být zahájeny, jakmile existuje vymezení programu, a mají být dokončeny jako součást analýzy požadavků a 
        mají se neustále rozšiřovat do větších úrovní podrobnosti, 
        jakmile budou k dispozici data, aby bylo možné činit rozhodnutí a provádět optimalizace nákladů a přínosů.\cite{2}

    \section{Základní zapojení bloků}
        \subsubsection*{Sériové zapojení}
            Při poruše jedné komponenty dojde k poruše celého systému. 
            Systém je v bezporuchovém stavu, pokud všechny jeho komponenty nemají poruchu.\cite{5}
            \begin{figure}[h]
                \centering
                \includegraphics[scale=0.75]{pic/seriove.png}
                \caption{Příklad sériového zapojení komponent} \label{Obrázek č. 2.1}
            \end{figure}

        \subsubsection*{Paralelní zapojení}
            K poruše celého systemu dochází pokud jsou v poruše všechny jeho komponenty. Bezporuchový stav trvá, dokud je alespoň jedna komponenta v bezporuchovém stavu.
            Z hlediska odhadu pravděpodobnosti představuje paralelní systém nejlepší variantu pro odhad pravděpodobnosti bezporuchového stavu.\cite{5}
            \begin{figure}[h]
                \centering
                \includegraphics[scale=0.75]{pic/paralelni.png}
                \caption{Příklad paralelního zapojení kopmonent} \label{Obrázek č. 2.1}
            \end{figure}

\chapter{Návrh desktopové aplikce .NET}
    \section{Objektová struktura}
        
    \section{Pomocné třídy}

\chapter{Průběh vývoje}
    \section{Rozdělení projektu na subprojekty}
        Celkové řešení je rozděleno na několik podprojektů, tak aby každý odpovídal svému účelu použití.
        \begin{itemize} 
        \item SpecianPRJ
        \item SpecianPRJ.Cli
        \item SpecianPRJ.Gui
        \item SpecianPRJ.Tests
        \end{itemize}
    \section{Zjednodušení diagramu}
        Pro zjednodušení tvorby diagramu bylo zavedeno pravidlo, že se systém skládá pouze ze série bloků, z nichž některé mohou reprezentovat paralelní zapojení.
        V programu nelze vytvořit diagram obsahující vazby mezi bloky, které spolu bezprostředně nesousedí.
\chapter{Testování}

    Pro testování funkčních bloků byla použita výchozí knihovna pro Unit testování v prostředí .NET pro desktopové aplikace MSTest.
    Za pomoci testování jsem došel ke správným výsledkům za pomoci připravené konfigurace a tím jsem ušetřil práci manuálním testováním.
    Další nespornou výhodou testování je odhalení chyb při změně tím,  že testovací metody odhalí neočekávané výledky.

    Testované byly třídy pro výpočet distribuční funkce.

\chapter{Návod k použití}

    Pro spuštění aplikace pro vývoj je potřeba mít nainstalované Visual Studio 2017 a novější. V přiloženém CD ve složce SpecianPRJ spusťte soubor SpecianPRJ.sln. 
    Pro standartní spuštění aplikace stačí otevřít soubor s příponou .exe.

    Pro obě varianty spuštní je nutným předpokladem nainstalovaný plný .NET Framework 4.6.1 a novější. 

    \section*{Založení nového diagramu}
    \section*{Uložení a otevření nového diagramu}
    \section*{Přidání prvku}
    \section*{Výpočty}

\chapter{Závěr}



\begin{thebibliography}{Mm99}
    \bibitem{1}
        28.6.2007, Prof. Ing. Václav Legát, DrSc., Zdroj: Verlag Dashöfer
    \bibitem{2}
        %https://theses.cz/id/dnvmwp/downloadPraceContent_adipIdno_11870
    \bibitem{3}
        http://gabben.wbs.cz/mtbf1.pdf
    \bibitem{4}
        IEEE 90
    \bibitem{5}
        %file:///C:/Users/King/Documents/PRJ/skripta/Plzen_06_10_2005.pdf
    \bibitem{6}
        https://homen.vsb.cz/~oti73/cdpast1/KAP03/PRAV3.HTM
    \bibitem{7}
        https://homen.vsb.cz/~oti73/cdpast1/KAP05/PRAV5.HTM
\end{thebibliography}

\end{document}
